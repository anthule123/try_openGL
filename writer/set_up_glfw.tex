\begin{enumerate}
    \item 
%phần 1: Cài trên visual studio code
\textbf{Cài trên visual studio code}


Cài theo \url{https://www.youtube.com/watch?v=Y4F0tI7WlDs} 
Để có được file glwr3.dll thì ta tải glwr bản binary.

Tham khảo \url{ https://github.com/Coddeus/opengl_cpp_vscode_sample }
file main.cpp và tasks.json.

\item
\textbf{Cài trên visual studio 2022}
\begin{itemize}
    \item Install Cmake
    \item Tải GLFW binary
    \item Lấy repo \url {https://github.com/JoeyDeVries/LearnOpenGL}
    rồi chạy file CMakeLists.txt trên visual studio 2022, 
    ấn build solution.
\end{itemize}
\end{enumerate}


\textbf{Một số thư viện}
\begin{enumerate}
    \item \textbf{Thư viện GLFW (Graphics Library Framework))} 
        \begin{itemize}
        \item GLFW là một thư viện mã nguồn mở được sử dụng để tạo cửa sổ
        và quản lý sự kiện đồ
        họa trong ứng dụng OpenGL.
        \item
        Nó cung cấp các chức năng cơ bản như tạo cửa sổ,
        xử lý sự kiện bàn phím và chuột, và quản lý vòng lặp chính
        cho ứng dụng OpenGL.
        \item
        GLFW là một thư viện nhẹ và đơn giản, thích hợp cho các dự án OpenGL 
        cơ bản và các ứng dụng đơn giản.
        \end{itemize}

     \item \textbf{Thư viện GLAD (Multi-Language GL/GLES/EGL/GLX/WGL
      Loader-Generator based on the official specs.)}
     
        OpenGL là một tiêu chuẩn hoặc một đặc tả đồ họa, không phải là một 
        phần mềm hoàn chỉnh được cài đặt sẵn trên máy tính. Thay vào đó,
        nó chỉ định cách các chức năng đồ họa cơ bản
        hoạt động và tương tác với card đồ họa. Để thực hiện việc vẽ đồ họa bằng
        OpenGL, bạn cần một trình điều khiển đồ họa (driver) cụ thể cho card đồ 
        họa bạn đang sử dụng.

        Mỗi nhà sản xuất card đồ họa có thể cung cấp một phiên bản OpenGL driver
        khác nhau dựa trên phiên bản card đồ họa của họ và các tính năng đặc biệt của nó. Điều này dẫn 
        đến việc có nhiều phiên bản khác nhau của driver OpenGL.

        Vấn đề là vì có nhiều phiên bản driver khác nhau, nên vị trí cụ thể
        của các hàm trong driver OpenGL không thể biết trước tại thời điểm
        biên dịch mã nguồn của bạn. Do đó, bạn cần phải truy vấn vị trí của 
        các hàm OpenGL này trong driver 
        tại thời điểm chạy (run-time).

        Điều này thực hiện thông qua thư viện như GLAD hoặc GLEW,
        làm cho việc truy cập các hàm OpenGL trở nên dễ dàng hơn.
        Bạn sẽ sử dụng các hàm như gladLoadGLLoader() để nạp các con trỏ
        hàm OpenGL và sau đó sử dụng các con trỏ này để gọi các chức năng
            OpenGL cụ thể trong mã của bạn.
\end{enumerate}

\textbf{Một số vấn đề}
\begin{enumerate}
    \item Làm sao để chạy nhiều file cpp trong vscode 
\end{enumerate}
