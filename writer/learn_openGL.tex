%tạo 1 article
\documentclass[12pt]{article}
\usepackage[utf8]{vietnam}
\usepackage{amsmath}
\usepackage{amsfonts}
\usepackage{graphicx}
\usepackage{float}
%lấy link
\usepackage{hyperref}
\usepackage{xcolor}
%tạo màu cho code

%lstlisting%nền code màu đen
\definecolor{mygray}{RGB}{50,50,50}
\definecolor{cyan}{RGB}{0, 255, 255}

\usepackage{listings}
\lstset{
    backgroundcolor = \color{mygray},
    basicstyle=\color{cyan},     % Màu văn bản trắng
    %đặt màu theo rgb
    keywordstyle=\color{red},    % Màu từ khóa đỏ
    commentstyle=\color{green},   % Màu chú thích (nếu bạn muốn đặt màu)
    stringstyle=\color{red},      % Màu chuỗi (nếu bạn muốn đặt màu)
    breakatwhitespace=false,      % Chia dòng chỉ khi khoảng trắng
    breaklines=true,              % Chia dòng tự động
    % Các cài đặt khác của listings theo ý muốn của bạn
    %chữ aria
}
\usepackage{caption}

\title{Thư viện openGL}
\author{anthule}
\date{\today}

\begin{document}
%tạo mục lục

\tableofcontents
\maketitle
\section{Giới thiệu}
OpenGL (Open Graphics Library) là một API đa nền tảng,
 đa ngôn ngữ cho render đồ họa vector 2D và 3D. API thường 
 được sử dụng để tương tác với bộ xử lý đồ họa (GPU), 
nhằm đạt được tốc độ render phần cứng

\begin {itemize}
\item API: Application Programming Interface: là một giao diện mà một hệ 
thống máy tính hay ứng dụng cung cấp để cho phép các yêu cầu dịch vụ 
có thể được tạo ra từ các chương trình máy tính khác, và/hoặc cho phép dữ liệu 
có thể được trao đổi qua lại giữa chúng. 
\item GPU: Graphics Processing Unit
Là bộ Vi xử lý chuyên phân tích những khối dữ liệu hình ảnh. 
GPU còn xử lý thông tin đa luồng, song song và bộ nhớ ở tốc độ cao.
 Kỹ thuật GPU đang dần trở nên dễ lập trình, cung cấp nhiều tiềm năng 
 cho việc tăng tốc xử lí cho nhiều chương trình với nhiều mục đích khác nhau,
  hơn cả chíp xử lí thông thường (CPUs).
  %chèn ảnh
\begin{figure}[ht]
\centering
\includegraphics[scale=0.5]{1112_cpu-va-gpu-1.jpg}
\end{figure}


\item render:
gọi tắt là \textbf{kết xuất}, là một quá trình kiến tạo một hình ảnh từ
 một
 mô hình (hoặc một tập hợp các mô hình) thành một cảnh phim 
hoặc hình ảnh nào đó bằng cách sử dụng phần mềm máy tính. 

\begin{itemize}
    \item Mô hình: là mô tả của các đối tượng ba chiều bằng một ngôn ngữ 
    được định nghĩa chặt chẽ hoặc bằng một cấu trúc dữ liệu. Mô tả này bao 
    gồm các thông tin về hình học, điểm nhìn, chất liệu và bố trí ánh sáng 
    của đối tượng. Hình ảnh này có thể là một hình ảnh số (digital image) 
    hoặc một hình ảnh đồ họa điểm (raster graphics image). Thuật ngữ này
     có
     thể tương đồng 
    với "quá trình một họa sĩ vẽ" một phong cảnh nào đấy.

    \end{itemize}


\end{itemize}
\section{Cách sử dụng OpenGL}
\subsection{Cài đặt}
\begin{enumerate}
    \item 
%phần 1: Cài trên visual studio code
\textbf{Cài trên visual studio code}


Cài theo \url{https://www.youtube.com/watch?v=Y4F0tI7WlDs} 
Để có được file glwr3.dll thì ta tải glwr bản binary.

Tham khảo \url{ https://github.com/Coddeus/opengl_cpp_vscode_sample }
file main.cpp và tasks.json.

\item
\textbf{Cài trên visual studio 2022}
\begin{itemize}
    \item Install Cmake
    \item Tải GLFW binary
    \item Lấy repo \url {https://github.com/JoeyDeVries/LearnOpenGL}
    rồi chạy file CMakeLists.txt trên visual studio 2022, 
    ấn build solution.
\end{itemize}
\end{enumerate}


\textbf{Một số thư viện}
\begin{enumerate}
    \item \textbf{Thư viện GLFW (Graphics Library Framework))} 
        \begin{itemize}
        \item GLFW là một thư viện mã nguồn mở được sử dụng để tạo cửa sổ
        và quản lý sự kiện đồ
        họa trong ứng dụng OpenGL.
        \item
        Nó cung cấp các chức năng cơ bản như tạo cửa sổ,
        xử lý sự kiện bàn phím và chuột, và quản lý vòng lặp chính
        cho ứng dụng OpenGL.
        \item
        GLFW là một thư viện nhẹ và đơn giản, thích hợp cho các dự án OpenGL 
        cơ bản và các ứng dụng đơn giản.
        \end{itemize}

     \item \textbf{Thư viện GLAD (Multi-Language GL/GLES/EGL/GLX/WGL
      Loader-Generator based on the official specs.)}
     
        OpenGL là một tiêu chuẩn hoặc một đặc tả đồ họa, không phải là một 
        phần mềm hoàn chỉnh được cài đặt sẵn trên máy tính. Thay vào đó,
        nó chỉ định cách các chức năng đồ họa cơ bản
        hoạt động và tương tác với card đồ họa. Để thực hiện việc vẽ đồ họa bằng
        OpenGL, bạn cần một trình điều khiển đồ họa (driver) cụ thể cho card đồ 
        họa bạn đang sử dụng.

        Mỗi nhà sản xuất card đồ họa có thể cung cấp một phiên bản OpenGL driver
        khác nhau dựa trên phiên bản card đồ họa của họ và các tính năng đặc biệt của nó. Điều này dẫn 
        đến việc có nhiều phiên bản khác nhau của driver OpenGL.

        Vấn đề là vì có nhiều phiên bản driver khác nhau, nên vị trí cụ thể
        của các hàm trong driver OpenGL không thể biết trước tại thời điểm
        biên dịch mã nguồn của bạn. Do đó, bạn cần phải truy vấn vị trí của 
        các hàm OpenGL này trong driver 
        tại thời điểm chạy (run-time).

        Điều này thực hiện thông qua thư viện như GLAD hoặc GLEW,
        làm cho việc truy cập các hàm OpenGL trở nên dễ dàng hơn.
        Bạn sẽ sử dụng các hàm như gladLoadGLLoader() để nạp các con trỏ
        hàm OpenGL và sau đó sử dụng các con trỏ này để gọi các chức năng
            OpenGL cụ thể trong mã của bạn.
\end{enumerate}

\textbf{Một số vấn đề}
\begin{enumerate}
    \item Làm sao để chạy nhiều file cpp trong vscode 
\end{enumerate}

\subsection{Tạo cửa sổ}
\subsubsection{Theo kiểu cũ - ở LearnOpenGL}
\begin{enumerate}
    \item Tại thư mục src, tạo file main.cpp.
    \item Tại dòng đầu của file main.cpp, gọi thư viện glad.h (glad trước glwr
     thì mới gọi ra được con trỏ các hàm của glwr), rồi gọi thư viện glwr.h.
     %hiển thị code, nền đen

    \begin{lstlisting}[language=C++]
        #include <glad/glad.h>
        #include <GLFW/glfw3.h>
    \end{lstlisting}

    \item \textbf{glfwInit()}:

    Việc sử dụng ${\#include <GLFW/glfw3.h>}$ trong mã nguồn của bạn là để
     bao gồm các khai báo và định nghĩa liên quan đến GLFW. Tuy nhiên,
      việc đưa mã nguồn của thư viện vào mã của bạn chỉ
     giúp bạn truy cập các khai báo và hàm từ GLFW mà bạn có thể gọi 
     trong mã của mình. Nó không thực sự khởi tạo hoặc quản lý bất kỳ
      thành phần cụ thể nào của GLFW cho bạn.

      glfwInit() là một hàm cụ thể của GLFW được sử dụng để khởi tạo thư 
      viện GLFW và chuẩn bị môi trường đồ họa. Khi bạn gọi glfwInit(), nó 
      thực sự thực hiện công việc khởi tạo thư viện và cài đặt các
       tài nguyên cần thiết cho việc làm việc với đồ họa.  

    \item \textbf{glfwWindowHint()}:
 
    Hàm glfwWindowHint trong thư viện GLFW được sử dụng để đặt các 
    cài đặt cho cửa sổ đồ họa mà bạn sẽ tạo sau đó bằng hàm glfwCreateWindow.
     Dưới đây là một số cài đặt phổ biến mà bạn có thể sử dụng với glfwWindowHint:
    
\begin{itemize}
    \item \texttt{glfwWindowHint(GLFW\_CONTEXT\_VERSION\_MAJOR, majorVersion)}
      và \\
      \texttt{glfwWindowHint(GLFW\_CONTEXT\_VERSION\_MINOR, minorVersion)}:
      
      Đặt phiên bản của OpenGL bạn muốn sử dụng. Ví dụ, để đặt phiên bản 
      OpenGL 4.5, bạn có thể sử dụng 
     \begin{lstlisting}[language=C++]
        glfwWindowHint(GLFW_CONTEXT_VERSION_MAJOR, 4);
        glfwWindowHint(GLFW_CONTEXT_VERSION_MINOR, 5);
     \end{lstlisting}
     \item \texttt{glfwWindowHint(GLFW\_OPENGL\_PROFILE,
      GLFW\_OPENGL\_CORE\_PROFILE)}:
      Đặt hồ sơ OpenGL bạn muốn sử dụng. Hồ sơ \texttt{ GLFW\_OPENGL\_CORE\_PROFILE 
      }sử dụng phiên bản core của OpenGL, 
      trong khi \texttt{GLFW\_OPENGL\_COMPAT\_PROFILE} sử dụng phiên bản tương thích
       ngược.
      %code 
        \begin{lstlisting}[language=C++]
            glfwWindowHint(GLFW_OPENGL_PROFILE, GLFW_OPENGL_CORE_PROFILE);
        \end{lstlisting}
    \end{itemize}

   \item \textbf {glfwCreateWindow()}:
   %code GLFWwindow* window = glfwCreateWindow(800, 600, "LearnOpenGL", NULL, NULL);
   \begin{lstlisting}[language=C++]
    GLFWwindow* window = glfwCreateWindow(800, 600, "LearnOpenGL", NULL, NULL);
   \end{lstlisting}
   Hàm glfwCreateWindow yêu cầu hai đối số đầu tiên lần lượt 
   là chiều rộng và chiều cao của cửa sổ. Đối số thứ ba 
   cho phép bạn đặt tên cho cửa sổ; hiện tại, bạn gọi nó là 
   "LearnOpenGL" nhưng bạn có thể đặt tên tùy ý. Bạn có thể bỏ 
   qua hai đối số cuối cùng. Hàm trả về một đối tượng GLFWwindow mà
    sau này bạn sẽ cần cho các hoạt động GLFW khác. Sau đó, bạn chỉ 
    định cho GLFW làm cho ngữ cảnh của cửa sổ của 
   bạn trở thành ngữ cảnh chính trên luồng hiện tại.
   \item Giữ cửa sổ ko bị đóng lại:
    %code
    \begin{lstlisting}[language=C++]
        while(!glfwWindowShouldClose(window))
        {
            glfwSwapBuffers(window);
            glfwPollEvents();    
        }
    \end{lstlisting}
      \begin{itemize}
        \item while (!glfwWindowShouldClose(window)): Đây là một
         vòng lặp chạy liên tục cho đến khi biểu thức
          glfwWindowShouldClose(window) trả về giá trị true.
           Hàm này kiểm tra xem GLFW đã được hướng dẫn để đóng cửa sổ hay 
           chưa. Nếu cửa sổ được đóng, biểu thức này trả về true và vòng 
        lặp dừng chạy, sau đó bạn có thể đóng ứng dụng.
        \item glfwSwapBuffers(window): Hàm này thực hiện việc hoán đổi
         (swap) hai bộ đệm màu. GLFW sử dụng hai bộ đệm màu để hiển 
         thị hình ảnh trên cửa sổ. Một bộ đệm được vẽ và lắng nghe dữ
          liệu đồ họa mới (color buffer), trong khi bộ đệm còn lại 
          hiển thị nội dung đã vẽ. Khi bạn gọi glfwSwapBuffers(window),
           nó sẽ chuyển đổi giữa hai bộ đệm, làm cho nội dung đã vẽ 
           xuất hiện trên cửa sổ.
        \item
        glfwPollEvents(): Hàm này kiểm tra xem có sự kiện nào được 
        kích hoạt (như sự kiện nhập từ bàn phím, chuyển động chuột, 
        vv.), cập nhật trạng thái của cửa sổ và gọi các hàm tương ứng 
        (các hàm được đăng ký thông qua các phương thức gọi lại callback).
         Nó giúp bạn xử lý sự kiện và tương tác người dùng trong ứng dụng
          của bạn.
        \item Một số dòng code khác: ko có xài thì cũng vẫn hiện cửa sổ lên.
        \begin{itemize}
            \item \texttt{glViewport(0, 0, 800, 600)}: Đặt kích thước
            \item   
            \texttt{glClearColor(0.2f, 0.3f, 0.3f, 1.0f)}: Đặt màu nền
            \item \texttt{glClear(GL\_COLOR\_BUFFER\_BIT)}: Xóa màu nền
            \item \texttt{glfwTerminate()}: Đóng GLFW
            \item \texttt{glfwSetFramebufferSizeCallback(window,
             framebuffer\_size\_callback)}: Đăng kí hàm callback 
                để khi thay đổi kích thước cửa sổ thì cửa sổ vẫn hiện lên
        \end{itemize}
        \item Tạo 2 cửa sổ: 
        %code
        \begin{lstlisting}[language=C++]
    GLFWwindow* window1 = glfwCreateWindow(width1, height1, 
    "Cua so 1", NULL, NULL);
    if (window1 == NULL) {
        glfwTerminate();
        return -1;
    }

    GLFWwindow* window2 = glfwCreateWindow(width2, height2, 
    "Cua so 2", NULL, NULL);
    if (window2 == NULL) {
        glfwTerminate();
        return -1;
    }
    glfwMakeContextCurrent(window1);
    glfwMakeContextCurrent(window2);

    \end{lstlisting}
    %chèn ảnh
    \begin{figure}[H]
        \centering
        \includegraphics[scale=0.5]{2_windows.JPG}
    \end{figure}
        \item Dòng code xử lí lỗi:
        \begin{itemize}
            \item Lỗi ko tạo được cửa sổ %code 
            \begin{lstlisting}[language=C++]
                if (window == NULL)
                {
                    std::cout << "Failed to create GLFW window" << std::endl;
                    glfwTerminate();
                    return -1;
                }
            \end{lstlisting}
            \item Lỗi ko tạo được ngữ cảnh %code
            \begin{lstlisting}[language=C++]
    glfwMakeContextCurrent(window);
    if (!gladLoadGLLoader((GLADloadproc)glfwGetProcAddress))
    {
        std::cout << "Failed to initialize GLAD" << std::endl;
        return -1;
    }
    \end{lstlisting}
            
\end{itemize}
        
        \end{itemize}
        
        
\end{enumerate}

\subsubsection{Theo kiểu mới - refacted code}

Tạo 1 file tên là \texttt{standard\_includes.cpp} trong thư mục include.
File này sẽ chứa các hàm init\_glfw chứa các hàm con để tạo cửa sổ, tạo ngữ cảnh, tạo vòng lặp 
để giữ cửa sổ ko bị đóng lại, xử lí lỗi, đóng GLFW, ..

Sau đó trong hàm main ta gọi hàm init\_glfw ra và tập trung vào
 việc vẽ hình.
\subsection{Vẽ một tam giác}

\end{document}
